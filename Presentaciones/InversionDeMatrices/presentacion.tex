\documentclass{beamer}
\usetheme{Warsaw}

\title{Inversión de matrices}
\subtitle{Para la solución de sistemas de ecuaciones lineales}
\author{Alan Cerón Chávez}

\institute{FES Acatlán - UNAM}
\date{\today}


\begin{document}
    \begin{frame}
        \titlepage
    \end{frame}

    \begin{frame}
        \frametitle{Contenido}
        \tableofcontents[sections=1-2]
    \end{frame}

    \begin{frame}
        \frametitle{Contenido}
        \tableofcontents[sections=3-5]
    \end{frame}

    \section{Introducción}
    \subsection{¿Qué es una matriz?}

    \begin{frame}
        \frametitle{¿Qué es una matriz?}
        Una matriz es un arreglo rectangular de números.
        $$
            \begin{bmatrix}
                a_{11} & a_{12} & \cdots & a_{1n} \\
                a_{21} & a_{22} & \cdots & a_{2n} \\
                \vdots & \vdots & \ddots & \vdots \\
                a_{1m} & a_{2m} & \cdots & a_{nm} \\
            \end{bmatrix}
        $$
    \end{frame}


    \subsection{¿Cómo se relaciona con un sistema de ecuaciones lineales?}
    \begin{frame}
        \frametitle{Para qué necesitamos invertir matrices}
        Podemos representar a un sistema de ecuaciones lineales como:
        $$
        Ax = b
        $$
    \end{frame}
    \section{Desarrollo}
    \subsection{La inversa de una matriz}
    \subsection{El detrminante de una matriz}
    \subsection{La matriz menor}
    \subsection{Co-factores}
    \subsection{Determinante para matrices n > 3}
    \subsection{Matriz de cofactores ó adjunta}
    \subsection{Inversión de matrices}
    \section{Ejercicios}
    \section{implementación}
    \section{Conclusiones}

    \begin{frame}
        \frametitle{Desarrollo}
        Sea $A$ una matriz y su determinante es distinta de cero, entonces:
        $$
        A^{-1} = \frac{1}{det(A)}adj(A)
        $$
        $$
        A^{-1} = \frac{1}{det(A)}C^{T}
        $$

    \end{frame}
\end{document}
