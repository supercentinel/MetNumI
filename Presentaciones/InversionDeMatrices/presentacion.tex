\documentclass{beamer}
\usepackage{amsmath}
\usetheme{Warsaw}

\title{Inversión de matrices}
\subtitle{Para la solución de sistemas de ecuaciones lineales}
\author{Alan Cerón Chávez}

\institute{FES Acatlán - UNAM}
\date{\today}


\begin{document}
    \begin{frame}
        \titlepage
    \end{frame}

    \begin{frame}
        \frametitle{Contenido}
        \tableofcontents[sections=1-2]
    \end{frame}

    \begin{frame}
        \frametitle{Contenido}
        \tableofcontents[sections=3-5]
    \end{frame}

    \section{Introducción}
    \subsection{¿Qué es una matriz?}

    \begin{frame}
        \frametitle{¿Qué es una matriz?}
        Una matriz $A$ es un arreglo rectangular de números de mn elementos, ordenados en m filas y n columnas.
        $$
        A=
            \begin{pmatrix}
                a_{11} & a_{12} & \cdots & a_{1n} \\
                a_{21} & a_{22} & \cdots & a_{2n} \\
                \vdots & \vdots & \ddots & \vdots \\
                a_{1m} & a_{2m} & \cdots & a_{nm} \\
            \end{pmatrix}
        $$
    \end{frame}


    \subsection{¿Cómo se relaciona con un sistema de ecuaciones lineales?}
    \begin{frame}
        \frametitle{Representación de un sistema de ecuaciones lineales como una matriz}
        sea el siguiente sistema de ecuaciones lineales:
        $$
        \begin{cases}
            2x + 3y = 8 \\
            3x + 4y = 9
        \end{cases}
        $$
        Los coeficientes de las variables se representan como una matriz:
        $$
        \begin{pmatrix}
            2 & 3 \\
            3 & 4
        \end{pmatrix}
        $$
        Y el sistema de ecuaciones lineales se puede representar de la manera $Ax=b$
        $$
        \begin{pmatrix}
            2 & 3 \\
            3 & 4
        \end{pmatrix}
        \begin{pmatrix}
            x \\
            y
        \end{pmatrix}
        =
        \begin{pmatrix}
            8 \\
            9
        \end{pmatrix}
        $$

    \end{frame}
    \subsection{Resolviendo el sistema de ecuaciones lineales}

    \begin{frame}
        Con la representación $Ax=b$ podemos resolver el sistema de ecuaciones lineales de la siguiente manera:
        \begin{align*}
            Ax &= b \\
            A^{-1}Ax &= A^{-1}b \\
            Ix &= A^{-1}b \\
            x &= A^{-1}b
        \end{align*}
    \end{frame}


    \section{Desarrollo}
    \subsection{¿Por qué?}

    \begin{frame}
        La matriz identidad $I$ es una matriz cuadrada que tiene unos en su diagonal principal y ceros en el resto de sus elementos.
        Tiene la propiedad de que al multiplicar cualquier matriz $A$ por $I$ el resultado es la misma matriz $A$.
        $$
        I =
        \begin{pmatrix}
            1 & 0 & \cdots & 0 \\
            0 & 1 & \cdots & 0 \\
            \vdots & \vdots & \ddots & \vdots \\
            0 & 0 & \cdots & 1 \\
        \end{pmatrix}
        $$

    \end{frame}


    \subsection{La inversa de una matriz}

    \begin{frame}
        \frametitle{Inversa de una matriz}
        El producto de una matriz $A$ por su inversa $A^{-1}$ es la matriz identidad $I$.
        Sea $A$ una matriz y su determinante es distinta de cero, entonces:
        $$
        A^{-1} = \frac{1}{det(A)}adj(A)
        $$
        $$
        A^{-1} = \frac{1}{det(A)}C^{T}
        $$
    \end{frame}

    \subsection{El detrminante de una matriz}
    \subsection{La matriz menor}
    \subsection{Co-factores}
    \subsection{Determinante para matrices n > 3}
    \subsection{Matriz de cofactores ó adjunta}
    \subsection{Inversión de matrices}
    \section{Ejercicios}
    \section{implementación}
    \section{Conclusiones}
    \subsection{Complejidad computacional}

    \begin{frame}
        \frametitle{Complejidad computacional}
        \begin{itemize}
            \item El calculo de la inversa de una matriz es un proceso que requiere de mucho tiempo de cómputo.\\
            \item La complejidad computacional de este proceso es de $O(n^3)$\\
            \item Por lo que es necesario implementar algoritmos que nos permitan obtener la inversa de una matriz de manera eficiente.\\
        \end{itemize}
    \end{frame}

    \subsection{Conclusiones}
    \begin{frame}
        \frametitle{Conclusiones}
        \begin{itemize}
            \item El calculo de la inversa de una matriz nos permite resolver sistemas de ecuaciones lineales.\\
            \item Su obtención es un proceso tedioso y que requiere de mucho tiempo de cómputo.\\
            \item Por lo que es necesario implementar algoritmos que nos permitan obtener la inversa de una matriz de manera eficiente.\\
            \item Aunque existen metodos para resolver sistemas de ecuaciones lineales que no requieren de la inversa de una matriz.\\
            \item En las proximas presentaciones se verán algunos de estos métodos.
        \end{itemize}
    \end{frame}

\end{document}
