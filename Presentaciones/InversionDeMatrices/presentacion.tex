\documentclass{beamer}
\usepackage{graphicx}
\usepackage{hyperref}
\usepackage{amsmath}
\usetheme{Warsaw}

\graphicspath{ {./images/} }

\title{Inversión de matrices}
\subtitle{Para la solución de sistemas de ecuaciones lineales}
\author{Alan Cerón Chávez}

\institute{FES Acatlán - UNAM}
\date{\today}


\begin{document}
    \begin{frame}
        \titlepage
    \end{frame}

    \begin{frame}
        \frametitle{Contenido}
        \tableofcontents[sections=1-2]
    \end{frame}

    \begin{frame}
        \frametitle{Contenido}
        \tableofcontents[sections=3-5]
    \end{frame}

    \section{Introducción}
    \subsection{¿Qué es una matriz?}

    \begin{frame}
        \frametitle{¿Qué es una matriz?}
        Una matriz $A$ es un arreglo rectangular de números de mn elementos, ordenados en m filas y n columnas.
        $$
        A=
            \begin{pmatrix}
                a_{11} & a_{12} & \cdots & a_{1n} \\
                a_{21} & a_{22} & \cdots & a_{2n} \\
                \vdots & \vdots & \ddots & \vdots \\
                a_{1m} & a_{2m} & \cdots & a_{nm} \\
            \end{pmatrix}
        $$
    \end{frame}


    \subsection{¿Cómo se relaciona con un sistema de ecuaciones lineales?}
    \begin{frame}
        \frametitle{Representación de un sistema de ecuaciones lineales como una matriz}
        sea el siguiente sistema de ecuaciones lineales:
        $$
        \begin{cases}
            2x + 3y = 8 \\
            3x + 4y = 9
        \end{cases}
        $$
        Los coeficientes de las variables se representan como una matriz:
        $$
        \begin{pmatrix}
            2 & 3 \\
            3 & 4
        \end{pmatrix}
        $$
        Y el sistema de ecuaciones lineales se puede representar de la manera $Ax=b$
        $$
        \begin{pmatrix}
            2 & 3 \\
            3 & 4
        \end{pmatrix}
        \begin{pmatrix}
            x \\
            y
        \end{pmatrix}
        =
        \begin{pmatrix}
            8 \\
            9
        \end{pmatrix}
        $$

    \end{frame}
    \subsection{Resolviendo el sistema de ecuaciones lineales}

    \begin{frame}
        Con la representación $Ax=b$ podemos resolver el sistema de ecuaciones lineales de la siguiente manera:
        \begin{align*}
            Ax &= b \\
            A^{-1}Ax &= A^{-1}b \\
            Ix &= A^{-1}b \\
            x &= A^{-1}b
        \end{align*}
    \end{frame}


    \section{Desarrollo}
    \subsection{¿Por qué?}

    \begin{frame}
        La matriz identidad $I$ es una matriz cuadrada que tiene unos en su diagonal principal y ceros en el resto de sus elementos.
        Tiene la propiedad de que al multiplicar cualquier matriz $A$ por $I$ el resultado es la misma matriz $A$.
        $$
        I =
        \begin{pmatrix}
            1 & 0 & \cdots & 0 \\
            0 & 1 & \cdots & 0 \\
            \vdots & \vdots & \ddots & \vdots \\
            0 & 0 & \cdots & 1 \\
        \end{pmatrix}
        $$

    \end{frame}


    \subsection{La inversa de una matriz}

    \begin{frame}
        \frametitle{Inversa de una matriz}
        El producto de una matriz $A$ por su inversa $A^{-1}$ es la matriz identidad $I$.
        Sea $A$ una matriz y su determinante es distinta de cero, entonces:
        $$
        A^{-1} = \frac{1}{det(A)}adj(A)
        $$
        $$
        A^{-1} = \frac{1}{det(A)}C^{T}
        $$
    \end{frame}

    \subsection{El detrminante de una matriz}

    \begin{frame}
        \frametitle{Caso 1: Matriz de 1x1}
        Sea $A$ una matriz de 1x1, entonces su determinante es el elemento de la matriz.
        $$
        A =
        \begin{pmatrix}
            a_{11}
        \end{pmatrix}
        \quad \det{A} = a_{11}
        $$
    \end{frame}

    \begin{frame}
        \frametitle{Caso 2: Matriz de 2x2}
        Sea $A$ una matriz de 2x2, entonces su determinante es:
        $$
        A =
        \begin{pmatrix}
            a_{11} & a_{12} \\
            a_{21} & a_{22}
        \end{pmatrix}
        \quad \det{A} = a_{11}a_{22} - a_{12}a_{21}
        $$
    \end{frame}

    \begin{frame}
        Sea $A$ una matriz de 3x3, entonces su determinante es:
        $$
        A =
        \begin{pmatrix}
            a_{11} & a_{12} & a_{13} \\
            a_{21} & a_{22} & a_{23} \\
            a_{31} & a_{32} & a_{33}
        \end{pmatrix}
        \quad \det{A} =
        \begin{align*}
            a_{11}(a_{22}a_{33} &- a_{23}a_{32}) \\
            &- \\
            a_{12}(a_{21}a_{33} &- a_{23}a_{31}) \\
            &+ \\
            a_{13}(a_{21}a_{32} &- a_{22}a_{31})
        \end{align*}
        $$
        O de la siguiente manera:
        $$
        \det{A} = a_{11}\begin{vmatrix}a_{22} & a_{23}\\ a_{32} & a_{33}\end{vmatrix} - a_{12}\begin{vmatrix}a_{21} & a_{23}\\ a_{31} & a_{33}\end{vmatrix} + a_{13}\begin{vmatrix}a_{21} & a_{22}\\ a_{31} & a_{32}\end{vmatrix}
        $$
    \end{frame}

        \begin{frame}
            \frametitle{Caso n: Matriz de nxn}
            Sea $A$ una matriz de nxn, entonces su determinante es:
            $$
            \det{A} = \sum_{i=1}^{n} a_{11}A_{11} + a_{12}A_{12} + \cdots + a_{1n}A_{1n}
            $$
        \end{frame}

    \subsection{Co-factores}

    \begin{frame}
        \begin{center}
            Qué es A{ij}?
        \end{center}\\
        A{ij} es llamado el cofactor de la matriz A en $ij$, y se define como:
        $$
        A_{ij} = (-1)^{i+j}\det{M_{ij}}
        $$

        Donde $M_{ij}$ es la matriz menor de $A$ en $ij$.\\
        Nota: $(-1)^{i+j}$ es el signo del cofactor. Y puede ser expresado como:
        $$
        (-1)^{i+j} =
        \begin{cases}
            1 & \text{si } i+j \text{ es par}\\
            -1 & \text{si } i+j \text{ es impar}
        \end{cases}
        $$
    \end{frame}

    \subsection{La matriz menor}

    \begin{frame}
        \frametitle{Qué es Mij?}
        Sea $A$ una matriz de nxn, entonces su matriz menor $M_{ij}$ de dimensiones (n-1)x(n-1) es la matriz resultante de eliminar la fila $i$ y la columna $j$ de la matriz $A$.\\
        Ejemplo:
        $$
        A =
        \begin{pmatrix}
            1 & 2 & 3 \\
            4 & 5 & 6 \\
            7 & 8 & 9
        \end{pmatrix}
        \quad M_{12} =
        \begin{pmatrix}
            4 & 6 \\
            7 & 9
        \end{pmatrix}
        $$
    \end{frame}

    \subsection{Matriz de cofactores}

    \begin{frame}
        \frametitle{Matriz de cofactores}
        Con todo lo anterior podemos definir la matriz de cofactores y la adjunta de una matriz.\\
        Comenzando con la matriz de cofactores:\\
        La Matriz de cofactores $C$ de una matriz $A$ es la matriz que contiene los cofactores de $A$.
        $$
        C = \begin{pmatrix}
            A_{11} & A_{12} & \cdots & A_{1n} \\
            A_{21} & A_{22} & \cdots & A_{2n} \\
            \vdots & \vdots & \ddots & \vdots \\
            A_{n1} & A_{n2} & \cdots & A_{nn}
        \end{pmatrix}
        $$
    \end{frame}

    \begin{frame}
        \frametitle{Matriz adjunta}
        La matriz adjunta es la matriz de cofactores transpuesta.
        $$
        adj(A) = C^T
        $$
    \end{frame}

    \subsection{Inversión de matrices}

    \begin{frame}
        \frametitle{Inversión de matrices}
        Finalmente, podemos definir la inversa de una matriz.\\
        Regresamos a la definición del inicio:
        $$
        A^{-1} = \frac{1}{\det{A}}adj(A)
        $$
        Y Si tenemos una matriz de $4x4$? Cuantas operaciones tenemos que hacer?\\
        \begin{center}
            ¿Para $n x n$?
        \end{center}
    \end{frame}
    \section{Ejercicios}
    \subsection{Ejercicio 1}

    \begin{frame}
        \frametitle{Matriz de 4x4}
        Sea la matriz $A$ de 4x4:
        $$
        A =
        %nmms copilot haz algo más complejo
        % \begin{pmatrix}
        %     1 & 2 & 3 & 4 \\
        %     5 & 6 & 7 & 8 \\
        %     9 & 10 & 11 & 12 \\
        %     13 & 14 & 15 & 16
        % \end{pmatrix}
        \begin{pmatrix}
            25 & 36 & 57 & -12 \\
            54 & 23 & 12 & 45 \\
            12 & 45 & 78 & 96 \\
            45 & 78 & 96 & 12
        \end{pmatrix}
        $$
        Calcular la inversa de $A$.\\
        $$
        A^{-1} = \frac{1}{\det{A}}adj(A)
        $$
    \end{frame}

    \subsection{Ejercicio 2}

    \begin{frame}
        \frametitle{Matriz de 8x8}
        Sea la matriz $A$ de 8x8:
        $$
        A =
        %Matriz de 8x8 con determinante diferente de 0
        \begin{pmatrix}
              3 & 2 & 1 & 4 & 0 & 5 & 6 & 7 \\
              0 & 8 & 9 & 0 & 1 & 0 & 2 & 0 \\
              7 & 0 & 3 & 0 & 6 & 2 & 0 & 8 \\
              4 & 0 & 5 & 9 & 0 & 1 & 0 & 0 \\
              0 & 6 & 0 & 0 & 7 & 0 & 9 & 0 \\
              1 & 0 & 0 & 0 & 8 & 4 & 0 & 3 \\
              0 & 0 & 4 & 5 & 2 & 0 & 7 & 1 \\
              2 & 0 & 0 & 3 & 0 & 9 & 8 & 6 \\
        \end{pmatrix}
        $$
        Calcular la inversa de $A$.\\
        $$
        A^{-1} = \frac{1}{\det{A}}adj(A)
        $$
    \end{frame}
    \subsection{Ejercicio 3}
        \begin{frame}
            \frametitle{Resolver el sistema de ecuaciones}
            Sea el sistema de ecuaciones:
            $$
            \begin{cases}
                12x + 3y - 4z = 17 \\
                -3x + 24y + 45z = -6 \\
                x + 9y - 6z = -13
            \end{cases}
            $$
            Resolver el sistema de ecuaciones utilizando la matriz inversa.
            $$
                A^{-1} = \frac{1}{\det{A}}adj(A)\qquad x = A^{-1}b
            $$
        \end{frame}

    \section{implementación}
    \subsection{QR al código}
    \begin{frame}
        \frametitle{Implementación en Rust}
        \begin{center}
            \includegraphics[width=170pt]{images/QR_rust_implementation.png}
        \end{center}
    \end{frame}
    \subsection{Live en Repl.it}

    \begin{frame}
        \frametitle{Live en Repl.it}
        \begin{center}
            \includegraphics[width=170pt]{images/QR_replit.png}
        \end{center}
    \end{frame}

    \section{Conclusiones}
    \subsection{Complejidad computacional}
    \subsubsection{¿Qué es la complejidad computacional?}

    \begin{frame}
        \frametitle{¿Qué es la complejidad computacional?}
        La complejidad computacional de un algoritmo es la cantidad de tiempo que tarda en ejecutarse. Independientemente
        de los recursos máquina en la que se ejecute.\\
        En este caso nos referimos a la cantidad de operaciones que se necesitan para obtener la inversa de una matriz.\\
        Si tenemos una matriz de $n x n$, cuántas operaciones tenemos que hacer?\\
        Para esto tenemos la notación $O()$ que calcula la complejidad computacional de un algoritmo en el peor caso.\\
    \end{frame}

    \subsubsection{Calculando la complejidad computacional para el determinante}

    \begin{frame}
        \frametitle{Calculando la complejidad computacional para el determinante}
        Debido a que el cálculo del determinante de una matriz es un proceso recursivo, podemos calcular su complejidad computacional de la siguiente manera:
        $$
        T(n) = T(n-1) + n
        $$
        Donde $n$ es el tamaño de la matriz.\\
        Y $T(n)$ es la complejidad computacional de la matriz de tamaño $n$.\\
        \begin{center}
            ¿Cuál es la complejidad computacional de la matriz de $n x n$?
        \end{center}
        La complejidad computacional de la matriz de $n x n$ es $O(n!)$
    \end{frame}

    \subsubsection{Calculando la complejidad computacional para la matriz adjunta}

    \begin{frame}
        \frametitle{Calculando la complejidad computacional para la matriz adjunta}
        De momento soy incapaz de explicar la complejidad computacional de la matriz adjunta.\\
        Pero de igual manera es un proceso recursivo.\\
    \end{frame}

    \begin{frame}
        \frametitle{Para concluir}
        Si queremos calcular $O()$ para $A^{-1}$ tenemos la suma complejidad del determinante y la adjunta; ambas de $O(n!)$.\\
        Y agregamos la complejidad del producto de una matriz por un escalar, que es de $O(n^2)$.\\
        \begin{itemize}
            \item El cálculo de la inversa de una matriz es un proceso que requiere de mucho tiempo de cómputo.\\
            \item La complejidad computacional de este proceso es de $O(n!)$!!!\\
            \item Por lo que es necesario implementar algoritmos que nos permitan obtener la inversa de una matriz de manera eficiente.\\
        \end{itemize}
    \end{frame}

    \subsection{Conclusiones}
    \begin{frame}
        \frametitle{Conclusiones}
        \begin{itemize}
            \item El cálculo de la inversa de una matriz nos permite resolver sistemas de ecuaciones lineales.\\
            \item Su obtención es un proceso tedioso y que requiere de mucho tiempo de cómputo.\\
            \item Por lo que es necesario implementar algoritmos que nos permitan obtener la inversa de una matriz de manera eficiente.\\
            \item Aunque existen metodos para resolver sistemas de ecuaciones lineales que no requieren de la inversa de una matriz.\\
            \item En las próximas presentaciones se verán algunos de estos métodos.
        \end{itemize}
    \end{frame}
    \subsection{Referencias}
    \begin{frame}
        Libros:
        \begin{itemize}
            \item Álgebra Lineal, Stanley I. Grossman, 6ta edición.
        \end{itemize}
        Páginas web:
        \begin{itemize}
            \item\href{https://learnxinyminutes.com/docs/asymptotic-notation}{Learn X in Y minutes Asymptotic Notation}
        \end{itemize}
        IA :
        \begin{itemize}
            \item Copilot para autocompletar la presentación y unas partes de la implementación.
            \item ChatGPT para Explicar la complejidad computacional.
        \end{itemize}
        Función para la lectura de matrices desde un archivo de texto:
        \begin{itemize}
        \item \href{https://github.com/robertfeliciano/linear-rustgebra/blob/main/src/lib.rs}{Usuario de Github: Robertfeliciano}
        \end{itemize}
    \end{frame}

\end{document}
